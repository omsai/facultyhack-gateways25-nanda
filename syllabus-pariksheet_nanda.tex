\documentclass[12pt]{article}
% Reduce excessive hyphenation https://stackoverflow.com/a/65156105
\tolerance=9999
\emergencystretch=10pt
\hyphenpenalty=10000
\exhyphenpenalty=100

\usepackage[
top=1cm,
bottom=1cm,
left=1cm,
right=1cm,
headsep=6pt,
headheight=17pt, % as per the warning by fancyhdr
includehead,
heightrounded, % to avoid spurious underfull messages
]{geometry}
\usepackage{fancyhdr}
\fancypagestyle{plain}{%
  \fancyhf{}                      % Clear header and footer fields.
  \fancyhead[l]{\today}
  \fancyhead[c]{Syllabus: Massively parallel mechanistic modeling}
  \fancyhead[r]{Page \thepage{} of~\pageref*{mylastpage}}}
\pagestyle{plain}
\usepackage{soul}
\usepackage{hyperref}           % clickable links
\hypersetup{colorlinks=true,citecolor=blue,linkcolor=blue,urlcolor=blue}
\urlstyle{rm}
\usepackage[
abbreviate=true,
bibencoding=utf8,
minnames=2,
maxbibnames=99,
sorting=none,
style=vancouver,
citestyle=numeric-comp
]{biblatex}
% The vancouver citation style is based on NLM per
% https://tex.stackexchange.com/a/371433
\addbibresource{references.bib}
\usepackage{upgreek}                % \upalpha
% Highlight author.
\renewcommand*{\mkbibnamegiven}[1]{%
  \ifitemannotation{highlight}{\textbf{\color{red} #1}}{#1}}
\renewcommand*{\mkbibnamefamily}[1]{%
  \ifitemannotation{highlight}{\textbf{\color{red} #1}}{#1}}
% Bibliography text size.
\renewcommand*{\bibfont}{\footnotesize}

\begin{document}

% https://poorvucenter.yale.edu/teaching/teaching-resource-library/syllabus-design



\begin{enumerate}
\item Strategies for within-node vectorization, caching, memory bandwidth, and
  I/O
  \begin{itemize}
  \item Exercises vectorizing Python code, inspecting C assembler CPU
    vectorized instructions, creating roofline plots to inspect CPU-memory,
    hwloc to inspect memory CPU hierarchy
  \item Goal is appreciating the value of low-level computing
  \end{itemize}
\item Strategies for task distribution and multi-node cluster scaling
  \begin{itemize}
  \item Exercises with C, TaskWorks
    (\url{https://github.com/hpc-io/taskworks}), Spack, MPI, and SLURM
  \item Goal is reducing I/O overhead by batching and threading
  \end{itemize}
\item Introduction to data parallel programming for GPUs
  \begin{itemize}
  \item Exercise with C++ Kokkos
  \end{itemize}
\item Pseudo-likelihood based parallel parameter estimation
  \begin{itemize}
  \item Exercise with Python pyABC ABC-SMC
  \end{itemize}
\item Introduction to performance engineering
  \begin{itemize}
  \item Exercise with Darashan, Drishti
    (\url{https://github.com/hpc-io/drishti-io}), and HPCtoolkit
  \end{itemize}
\item Stress-free peer code review and testing
  \begin{itemize}
  \item Exercise with professor and pairing
  \end{itemize}
\item Capstone project; potential topics for students without existing,
  relevant research projects:
  \begin{itemize}
  \item Scaling up data science-centric workflows (with CharlieCloud)
  \item On-demand complexity at scale (with Apptainer, Galaxy, and )
  \item AI acceleration (with biologically-informed neural networks [BINNs])
  \end{itemize}
\end{enumerate}

\label{mylastpage}              % chktex 24
\newpage
\appendix
\setcounter{page}{1}
\renewcommand{\thepage}{\arabic{page}}
\fancyhead[c]{References}
\fancyhead[r]{Appendix Page \thepage{} of~\pageref*{mylastappendixpage}}
\printbibliography[heading=none]{}%
\label{mylastappendixpage}

\end{document}
